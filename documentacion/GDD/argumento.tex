En este apartado detallaremos el argumento y el guión del juego. Al ser \nombrejuego un videojuego del género de las aventuras gráficas, estos tienen más importancia, pues a través de la narración vamos conociendo a los personajes y a cómo resolver los puzzles que nos deparará en el juego. 

    \section{Argumento y narrativa}
        \subsection{Trasfondo argumental}
        Hace un par de días, el Protagonista ha suspendido un examen con la infame nota de un 4,9. Ante tal nota, el Protagonista se dispone ir a la revisión del examen al despacho del profesor. Lamentablemente el profesor no se encuentra allí, pero cómo nuestro Protagonista es un cabezota, hará todo lo posible para encontrar al profesor y convencerle de que merece el aprobado.
        
        \subsection{Elementos clave}
        En esta sección es útil para definir los puntos importantes en el argumento a desarrollar durante la partida al videojuego. Lamentablemente, \nombrejuego tiene su argumento intrínsecamente ligado a los puzzles y escenarios que se incluirán en él. Como no están totalmente definidos a día de esa versión, no se puede hablar apenas de dichos elementos aquí.
        
      	De lo poco que se puede decir es lo siguiente: 
      	\begin{enumerate}
      		\item El personaje principal descubrirá que el profesor no está en su despacho y tiene que buscarlo.
      		\item El persona principal encontrará una nota que le llevará a la biblioteca.
      		\item La bibliotecaria le dice al protagonista (después de resolver el puzzle de la biblioteca) que la última vez que vio al profesor dijo que iba a la firma de un libro.
      		\item ...
      		\item El personaje principal encuentra al profesor compañero de nuestro profesor, este le dice que vuelva al despacho que seguramente esté ya allí.
      		\item El personaje principal vuelve al despacho y, efectivamente, el profesor se encuentra allí. Nos suplica ir con él a una junta directiva para demostrarles que sus métodos de enseñanza son válidos, o si no, le echan.
      		\item El personaje principal pasa la prueba de la junta directiva, estos y el profesor le dan su enhorabuena. Finalmente obtiene su aprobado. Obteniendo el final feliz del juego. 
      	\end{enumerate}
      	
        
        \subsection{Secuencias cinemáticas}
        En esta sección detallaremos las secuencias cinemáticas que habrá en \nombrejuego. Estos son momentos del juego en el que el \emph{Jugador} no podrá controlar al Protagonista y el juego tomará control de la situación. En juegos grandes se suele invertir una cantidad enorme en hacer las secuencias cinemáticas lo más vistosas posibles empleando en su mayoría vídeos. Juegos más pequeños prefieren evitar estas escenas por el enorme gasto que suponen, y se las ingenian en hacerlos de otras maneras.
        
        En \nombrejuego al ser también un juego de pequeña envergadura, la cantidad de secuencias cinemáticas del juego es mínima y sólo habrá una al principio de entrar en un nuevo escenario, y otra después de resolver un puzzle. 
        
        Por ahora solo hay un par de secuencias esbozadas ligeramente, las cuales son:
        \begin{itemize}
			\item \negrita{Inicio del juego}: Al iniciar el juego veremos como el protagonista camina por el pasillo de los despachos, murmurando para sí mismo sobre el último examen y la nota que ha obtenido. Entra en el despacho, y al no ver nadie, empieza a examinar el despacho para saber dónde puede haber ido el profesor.
			\item \negrita{Final del juego}: El profesor le da su enhorabuena y le confirma que por sus esfuerzos le ha el 5 que quería el alumno. Y que ya se verán las caras el próximo cuatrimestre.	
        \end{itemize}
            
    \section{Mundo del juego}
    Definir el mundo donde va a transcurrir un videojuego, sea real o ficticio, es vital para lograr la ambientación correcta que queramos transmitir al juego. Si bien es cierto que algunos juegos pueden permitirse el reducir al mínimo esta parte, cualquier juego que quiera introducirte una historia o un escenario que explorar, tendrá que decirte qué es ese sitio, por qué está ahí y cuáles son sus características.
    
    \nombrejuego transcurre en la Cádiz actual, de hecho la mayoría de los sitios a visitar durante el juego tendrán una correspondencia en la realidad. Sin embargo, estos poseerán algunas modificaciones en concordancia a las necesidades de los puzzles, las limitaciones gráficas, o la comicidad del juego. Por lo tanto estaremos ante una Cádiz actual satirizada, para ser más exactos. 
    
        \subsection{Estilo general}%General Look and Feel
        Todo lo que se narrará tanto en los diálogos como en las descripciones, se realizará en tono humorístico e irónico. Pues lo que se intentará retratar en el juego es un humor similar al de aventuras gráficas como Monkey Island, pero en tono más gamberro. 
        
        \subsection{Primer escenario}
        El despacho es el lugar donde el profesor está cuando no hay clases, allí redacta y corrige trabajos y exámenes. También hace tutorias a sus alumnos cuando estos quieren pedirle clemencia con las notas. 
        
        Habrá diversos objetos interactuables y diálogos que más adelante se colocarán aquí.
        \subsection{Segundo escenario}
        La biblioteca es donde se acumulan todos los libros relacionados con las filologías y estudios históricos de la universidad. Sitio de estudio tranquilo, aunque actualmente el silencio de la biblioteca se ve interrumpido con los cuchicheos de los estudiantes. La anciana bibliotecaria no puede vigilar a los alumnos y hacer sus quehaceres con la edad. Con el resultado de tener un caos con los periódicos del Cádiz de 1812 apilados en la entrada.
        
        Habrá diversos objetos interactuables y diálogos que más adelante se colocarán aquí.
        %...
        
    \section{Personajes}
    En esta sección definiremos brevemente a los personajes que intervendrán en el juego. Servirá sobretodo a la hora de definir las personalidades y reflejarlas en la manera de hablar de estos en el guión.
    
        \subsection{Protagonista}
            \subsubsection{Trasfondo}
            Estudiante del Grado de Historia en la Universidad de Cádiz, ha suspendido un examen con un 4,9 y hará todo lo posible para convencer al profesor de que merece el aprobado.
            \subsubsection{Personalidad}
            De personalidad humorística y despreocupada, con cierta aversión al profesor que le ha suspendido.
            \subsubsection{Aspecto}
            Pelo negro y desaliñado, con barbita dejada tal y cómo se lleva ahora entre los universitarios. Con sudadera y vaqueros.
                
        \subsection{Profesor}
            \subsubsection{Trasfondo}
            Profesor de una asignatura en el Grado de Historia en la Universidad de Cádiz, no se sabe donde está.
            \subsubsection{Personalidad}
            Despistado y afable, pero es muy estricto con los alumnos.
            \subsubsection{Aspecto}
            Típico profesor mayor con barba y traje.
            
        \subsection{Bibliotecaria}
            \subsubsection{Trasfondo}
            Es la bibliotecaria de la Facultad de Filosofía y Letras desde hace muchos años, ya los años les pasa factura y se olvida de las cosas o no puede llevar los libros con tanta facilidad.
            \subsubsection{Personalidad}
            Olvidadiza y muy estricta con las normas de la biblioteca, sobretodo con la de mantener el silencio.
            \subsubsection{Aspecto}
            Mujer mayor con gafas de lectura antigua.
    
    Estos son los personajes definidos por ahora, más adelante se irán añadiendo más y así sucesivamente hasta completar con todo el elenco de personajes que harán su aparición en el juego.
        %...