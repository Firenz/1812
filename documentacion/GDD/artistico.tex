En este apartado detallaremos el estilo que deben seguir los gráficos en el juego a partir del arte conceptual, las guías de estilo, descripciones de cómo deben ser personajes y escenarios, escenas cinemáticas, etc.

    \section{Arte conceptual}
    El arte conceptual, en un videojuego, sirve para definir tanto personajes, escenarios como el estilo de los gráficos de los que va a tratar el juego. Normalmente en juegos que pretenden dar un toque distintivo mediante el apartado gráfico, ya sea por diferenciar su estilo propio o para causar determinadas emociones o ambientación, es muy importante. Pero en cambio, en juegos pequeños o que los gráficos sean más secundarios, no suelen tener tanta carga y se deja más al gusto personal del grafista. \nombrejuego un juego medianamente pequeño, su arte conceptual es escaso y se deja también en manos de la grafista. 
    %El único arte conceptual que incluiremos es el diseño inicial del Protagonista, por dar un ejemplo:
    
    \section{Guías de estilo}
    En los videojuegos siempre es necesario que haya una uniformidad en los gráficos, no sólo con el estilo visual, sino con el tamaño de los gráficos y de las pantallas. Asimismo, \nombrejuego propondrá las siguientes guías para los gráficos y que así la grafista pueda atenerse a unas normas a la hora de la realización de los gráficos. Dichas normas, son más de sugerencia para conseguir una uniformidad que una imposición hacia el grafista. Así que si la ocasión lo requiere, el grafista podría saltarse las guías siempre y cuando se consulte con los otros miembros del equipo relacionados con el arte del juego y se llegue a un consenso para ese cambio. Estas son las guías para \nombrejuego:
    
    \begin{itemize}
    \item Personajes: 64x40 píxeles de resolución original, al que luego se le aplicará un \emph{zoom} por dos veces.
    \item Escenarios: 320x240 píxeles de resolución original, al que luego se le aplicará un \emph{zoom} por dos veces.
    \item Pantalla: 1280x960 píxeles será la resolución original de la pantalla, a partir de ella se adaptará a las distintas resoluciones más comunes en los ordenadores.
    \item Objetos: 32x32 píxeles de resolución original, al que luego se le aplicará un \emph{zoom} por dos veces.
    \item Paleta de colores: preferiblemente 256 colores (8 bits) para simular mejor el ser un juego antiguo, pero si se diera el caso de necesitar más colores, no habría ninguna restricción en usarlos.
    \end{itemize}
    
    \section{Personajes}
    En esta sección haremos una pequeña descripción de los personajes que aparecen en el juego, de forma que esta sea útil para la grafista a la hora de saber cómo diseñar a estos. Los personajes por ahora son los siguientes:
    
    \begin{itemize}
    \item \negrita{Protagonista}: Un estudiante universitario con los pelos revueltos, con sudadera, vaqueros y zapatillas.
    \item \negrita{Profesor}: El típico profesor mayor de universidad, con barba y traje.
    \item \negrita{Bibliotecaria}: La típica bibliotecaria, una señora mayor con gafas de lectura colgadas en la pechera.
    \item \negrita{Escritor}: Un escritor afrancesado y borracho, sentado en la mesa de firma de sus libros.
    \item \negrita{Junta directiva}: Varios hombres sentados con traje de ejecutivo y mirada inquisitoria, sentados en una mesa larga como si fuera un juicio.
    \end{itemize}
    
    \section{Escenarios}
    En esta sección haremos una pequeña descripción de los escenarios que aparecen en el juego, de forma que esta sea útil para la grafista a la hora de saber cómo diseñar a estos. Los escenarios por ahora son los siguientes:
    
    \begin{itemize}
    \item \negrita{Pasillo de la Facultad}: Un pasillo de la Facultad con varias puertas que dan a los despachos de distintos profesores.
    \item \negrita{Despacho del profesor}: El despacho al que pertenece el profesor con el que quiere hablar el Protagonista, tiene dos mesas (son dos profesores por despacho), una estanteria con libros y otras cosas, un tablón de corcho con un mapa, carteles, etc.
    \end{itemize}
    
    Estos son los escenarios por ahora, se irán añadiendo más conforme se avance el desarrollo del juego.
    
    \section{Objetos}
    En esta sección detallaremos brevemente los objetos que el Protagonista puede llevar en el inventario. Los objetos por ahora son los siguientes:
    \begin{itemize}
    \item \negrita{Examen suspendido}: Una hoja escrita con un 4,9 en rojo.
    \item \negrita{Banderitas}: Una banderita española y varias francesas con una chincheta.
    \item \negrita{Nota de la biblioteca}: Una pequeña nota que indica que el profesor debe ir a la biblioteca.
    \end{itemize}
    
    \section{Escenas cinemáticas}
    En \nombrejuego prácticamente no hay escenas cinemáticas, pues se hará uso de las animaciones de los personajes para crear las escenas controladas por el juego. Solo habrá una escena cinemática como tal, que será el final del juego en el que se verán imágenes del Protagonista feliz por haber aprobado y el de otros personajes a modo de epílogo, mientras se muestras los créditos y los agradecimientos.
    
    \section{Miscelánea}
    En esta sección suelen entrar los gráficos que no tienen cabida en las otras secciones. Concretamente en \nombrejuego se necesitarán los siguientes gráficos extras:
    \begin{itemize}
    \item \emph{Smartphone} del Protagonista, negro y únicamente táctil.
    \item Botones de las aplicaciones del \emph{smartphone} del Protagonista.
    \item Mano del Protagonista que hace como que toca el \emph{smartphone}.
    \end{itemize}