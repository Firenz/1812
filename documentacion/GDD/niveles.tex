%Los niveles en este juego, serán los distintos escenarios donde el Jugador pueda moverse. \nombrejuego tiene pocos niveles por las limitaciones de tiempo y dinero que implica hacer un proyecto final para una carrera universitaria, pero todos estos están esquematizados para esclarecer todo lo que contiene y sucede dentro de dicho nivel.

Los niveles son los distintos escenarios por los que un \emph{Jugador} tendrá que pasar para poder avanzar en el videojuego, de forma que \negrita{siempre} hay que resolver el problema que nos supone el nivel actual para poder llegar al siguiente, y así sucesivamente hasta que acabemos el juego. Estos niveles o escenarios pueden ser pantallas estáticas tales como un nivel del Tetris, o un escenario donde el \emph{Jugador} pueda moverse de un lado para otro. En este segundo caso pueden existir dos variantes, que el \emph{Jugador} pueda recorrer el nivel con una vista en primera persona, o en tercera persona teniendo el \emph{Jugador} que mover un personaje para recorrerlo.

En cualquier caso, el diseño de los niveles es \negrita{vital} para que el jugador disfrute de la experiencia de superar el videojuego en concreto que vayamos a crear. Hay que lograr que nunca haya momentos aburridos y a su vez que no haya demasiados momentos de tensión que acaben en frustración para el \emph{Jugador}, y que acaben consiguiendo que este lo deje y nunca termine el juego. De hecho, aunque un juego pueda poseer buenas e interesantes mecánicas, los \emph{Jugadores} no las verán hasta que entren en juego en algún nivel e interactuen con este.

%En el caso de aventuras gráficas, concretamente \nombrejuego que es el juego a desarrollar aquí, sus niveles van divididos en distintos escenarios a recorrer. En estos hay que buscar objetos o sitios donde poder interactuar con el escenario u otros personajes, de forma que si hacemos la combinación correcta, logramos resolver el problema que nos han propuesto y así seguir avanzando hacia nuestro objetivo. En resumen: 

En el caso de aventuras gráficas, concretamente \nombrejuego que es el juego a desarrollar aquí, sus niveles van divididos en distintos escenarios a recorrer. De forma concisa, estos son los pasos que hay que seguir en el diseño de los niveles en este tipo de género:

\begin{enumerate}
	\item Al inicio del juego, nos plantean en el argumento un problema u objetivo a conseguir.
	\item Para poder conseguir el objetivo principal, hay que dividirlo en distintos sub-objetivos.
	\item Cada sub-objetivo se resuelven con la combinación de acciones y objetos adecuadas.
	\item Una vez resuelto dicho sub-objetivo, obtenemos nuevos niveles o escenarios a visitar.
	\item En los escenarios nuevos nos proporcionarán un nuevo sub-objetivo que nos acercará al objetivo principal del videojuego.
	\item Se repite el proceso de sub-objetivos hasta que finalmente recompensa a los \emph{Jugadores} con la obtención del objetivo principal del juego.
\end{enumerate}

Como véis, en las aventuras gráficas, muchas veces hablar de puzzles es sinónimo de hablar de niveles. En un principio, en \nombrejuego se tomó la decisión inicial de separar la parte de los puzzes (niveles) en otro documento mejor organizado. No se descarta en un futuro incluir dicha parte en este documento, pero para la versión del documento de diseño que nos ocupa, este capítulo sólo se quedará como orientativo. 

Las secciones siguientes se explicarán de forma breve cuál es su propósito en el juego, y de cómo podríamos escribirlas.


    \section{Nivel 1}
    Este es el nivel donde comenzamos el juego. Suele ser más simple y largo que otros, pues se usa principalmente para explicarle al \emph{Jugador} de qué va el juego y cuál es el objetivo principal de este, además de proporcional diversos elementos con los que el \emph{Jugador} empiece a familiarizarse con las mecánicas básicas del juego. 
        \subsection{Sinopsis}
        Aquí vendría un resumen de lo que va a pasar en este nivel, de sus características y elementos más destacados.
        \subsection{Introducción}
        Aquí deberíamos escribir qué es lo que ha sucedido antes de empezar el nivel y los hechos que pueden influir en el transcurso de este.  
        \subsection{Objetivos}
        Esto es lo antes comentado con los sub-objetivos del juego, aquí hay que decir qué es lo que el \emph{Jugador} tiene que lograr hacer en este nivel.
        \subsection{Descripción física}
        Tal y como indica el nombre, aquí incluiríamos una descripción detallada del nivel y de sus elementos.
        \subsection{Mapa}
        En el caso de que el nivel fuera de gran extensión, o que el nivel perteneciera a un lugar concreto dentro del mundo en el que se va a desarrollar el juego, se tendrían que incluir aquí imágenes diversas explicando los puntos de referencia o elementos clave del nivel. Así tanto para que el que se encarga de diseñar los niveles tenga claro donde colocar los elementos dentro del nivel, como para que el \emph{Jugador} tenga puntos de referencia para orientarse dentro del nivel.
        \subsection{Camino crítico}
        En los niveles, pueden haber distintas maneras de obtener los elementos clave para conseguir el objetivo del nivel. Aquí habría que detallar todas las rutas diferentes que un \emph{Jugador} puede seguir a través del nivel para conseguir el objetivo. Aunque hay que puntualizar, que a veces las rutas de resolución de un nivel pueden ser casi infinitas, en ese caso, solo tendríamos que escribir aquí las rutas principales que usará la mayoría de \emph{Jugadores}. 
        \subsection{Resolución del nivel}
        Aquí, a modo de guía de resolución de juego, tenemos que explicar cómo se resuelve el problema que nos presenta el nivel para lograr el objetivo.
        \subsection{Conclusión}
        Aquí comentaremos los sucesos que ocurran una vez finalizado el nivel y antes de que comience el siguiente.
        
    %\section{Nivel 2}
    %...