Este es el documento de diseño de \nombrejuego. El cual es un videojuego en 2D del género de las aventuras gráficas, con una ambientación en la Cádiz actual cuyo objetivo es tanto de entretener al jugador como que aprenda anécdotas y hechos relacionados con Cádiz en los años que fue asediada y vieron nacer la Constitución de 1812.

    \section{Sinopsis}
        \nombrejuego es un videojuego en el que controlaremos a un estudiante de la Licenciatura de Historia de la Universidad de Cádiz, que recientemente ha suspendido un examen y va al despacho a pedir una revisión para su nota. Sin embargo, no logra encontrar al profesor y se embarcará en una aventura para descubrir el paradero de dicho profesor ayudándose de las pistas obtenidas al resolver puzzles, teniendo estos siempre una relación con la Cádiz de 1812. El videojuego tiene que estar completamente documentado, pues una gran gran parte de la jugabilidad y de la parte educativa del juego, recae completamente en el buen diseño que se haga de los puzzles, estancias e interacciones con otros personajes no controlables.
        
    \section{Género}
        \nombrejuego pretende seguir el patrón de las aventuras gráficas antiguas. Dicho género basa sus mecánicas en ir avanzando por el mundo, escenario o juego a través de la resolución de diversos puzzles, planteados como situaciones que se suceden en la historia, interactuando con personajes y objetos a través de un menú de acciones o interfaz similar, utilizando un cursor para manejar al personaje y realizar las distintas acciones.
        
    \section{Propósito y público objetivo}
        El principal objetivo de \nombrejuego es de proporcionar documentación en español sobre el diseño de videojuegos, y más específicamente en el campo de las aventuras gráficas con el documento de diseño de puzzles. También el de promocionar la historia de Cádiz. Finalmente, realizar el videojuego a partir de los diseños, poniendo a disposición de la gente el código documentado del juego y así promocionar el desarrollo organizado para proyectos con una mediana envergadura. No obstante, debe de ser un producto que sea jugable y ameno para que este desarrollo haya cumplido con sus objetivos.
        
        \nombrejuego está dirigido a todas las edades a partir de los seis años, pero principalmente a los jóvenes por ser un videojuego sencillo y tratado con un humor más juvenil. 
        
    \section{Resumen del flujo de juego}%Game flow summary
        Si bien el videojuego va a ser 2D, el movimiento del personaje dentro de los escenarios (dentro de las zonas donde se pueda caminar) será de arriba y abajo, derecha e izquierda y en diagonales. Luego para moverse entre los escenarios, se hará uso de un mapa con las localizaciones marcadas. Para poder avanzar en el juego será necesario resolver puzzles en los escenarios para poder acceder a nuevas localizaciones en el mapa.
        
    \section{Estilo visual}
    \nombrejuego tendrá un estilo sencillo, sin ser demasiado detallista para encajar con su carácter amigable y accesible. El estilo visual que más encaja sería el de las aventuras gráficas antiguas, las cuales eran gráficos pixelados asemejando a dibujos animados o cómics. Los personajes y escenarios serán caricaturescos, con colores vivos y trazos simples. En el caso de los escenarios, estos estarán basados en sitios reales a los que se les habrá simplificado o modificado para encajar con el estilo visual del videojuego.
        
    \section{Alcance del proyecto}%Project scope
    El objetivo principal es desarrollar un sistema de juego sólido, al que podamos ir añadiéndole contenidos sin apenas dificultad. En primera instancia el juego contará con nueve escenarios distintos, cada uno con uno o dos puzzles. Cada escenario representará un lugar distinto de Cádiz, y por consiguiente con una temática distinta tanto visualmente como en el problema histórico a tratar en el puzzle.
    
        \subsection{Número de niveles}
        En un principio son nueve niveles, pero este número puede variar con facilidad.
        
        \subsection{Número de Personajes No Jugables (PNJ o NPC en inglés)}
        Por ahora el número de PNJs definidos son unos cinco.
        
        \subsection{Número de puzzles}
        En un principio son diez puzzles, incluyendo un puzzle final que haga un repaso de todos las anécdotas históricas relatadas en el videojuego.
