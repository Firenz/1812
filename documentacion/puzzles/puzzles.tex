En este capítulo escribiremos todos los puzzles contenidos en \nombrejuego. Se dividirán en secciones centradas en cada escenario, pues cada uno tendrá un puzzle distinto. En estas secciones se escribirá sobre cómo se le plantea al \emph{Jugador} el puzzle, un mapa con los objetos necesarios o clave para obtener información de cómo resolverlos, y finalmente un camino de resolución del puzzle.

\section{Despacho}
\subsection{Tutorial introductorio}

\fbox{
	\parbox{\textwidth}{
		%\begin{center}
			\centerline{\bf Escena cinemática: Introducción}

			El alumno entra al despacho, llama al profesor pero ve que no hay nadie.\\
			
			``¿Hola? ¿Profesor?... No hay nadie. Supongo que esperaré a que llegue.''\\
			
			Se pone a mirar el móvil como haría cualquier universitario. Tras un momento deja el móvil y se queja de la temperatura en el despacho.\\
			
			``Buff. ¡Qué calor hace aquí dentro! Si voy a esperar mucho tiempo al menos abriré la ventana.''\\
			
			Entonces es cuando el \emph{Jugador} obtiene el control por primera vez.
		%\end{center}
	}
}

	\subsubsection{Objetivo} 
	Moverse hacia la ventana y abrirla con las indicaciones de los controles.
	
	-Haz click en la pantalla en la dirección en la que quieras moverte-
	/*El jugador hace click en cualquier parte y se mueve hacia allí*/
	-Si el cursor cambia de aspecto al pasarlo sobre un elemento del escenario es que puedes interactuar con él. Prueba a pasarlo por encima de la ventana-
	/*Al hacerlo el cursor cambia y salta otro mensaje*/
	-Ahora que sabes que puedes hacer algo con este elemento tienes dos opciones: con el click derecho del ratón puedes interactuar directamente con él o con el izquierdo puedes examinarlo primero. Examina la ventana y después ábrela.-
	/*El jugador debe examinar la ventana antes de poder abrirla para encontrar el pestillo de lo contrario le saltará este diálogo*/
	``Quiero abrirla pero no encuentro el pestillo. Debería mirarla más de cerca.''
	Examinar ventana
	``¿Dónde está el pestillo?... ¡Ah! Aquí está.''
	Ventana examinada => Usar ventana
	El personaje abre la ventana. Sopla el viento. Caen las banderitas del mapa colgado del tablón en la pared
	``¡Fantástico! Sólo me faltaría que ahora me acusaran de vandalismo por tirar las cosas del profesor. Será mejor que lo recoja.''
	
	[Recoge las banderitas]
	
	(Tras recoger las banderitas el personaje mira el tablón)
	``...''
	
	[Pausa]
	
	``¿Cómo estaba esto puesto?''
	
	[Pausa]
	
	``Agh… No hay manera. Tiene que ver alguna forma de saber dónde va cada bandera.''
	(Mira hacia la estantería del lado contrario al que estaba mirando)
	``Igual en alguno de estos libros…''
	
\subsection{Mapa del despacho}
	\subsubsection{Objetivo}	
	Pon las banderitas en sus posiciones correctas.

\section{Biblioteca}


\section{Café}


