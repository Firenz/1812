% -*-programas.tex-*-
% Este fichero es parte de la plantilla LaTeX para
% la realización de Proyectos Final de Carrera, protejido
% bajo los términos de la licencia GFDL.
% Para más información, la licencia completa viene incluida en el
% fichero fdl-1.3.tex

% Copyright (C) 2009 Pablo Recio Quijano 


%Es usual en un PFC referenciar que software has usado para la
%realización del mismo. Aprovecharé este apartado para que conozcas
%alguna herramienta que puede serte de ayuda para realizar tus
%documentos en \LaTeX{}

%\section*{Emacs + Auc\TeX}

%Emacs es uno de los programas de edición más usados por
%desarrolladores de software, ya que es bastante versatil admitiendo
%gran cantidad de ``plugins'' o extensiones que permiten ampliar aun
%más sus funcionalidades.\\

%Uno de estos plugins es Auc\TeX \cite{pdf:auctex}, el cual incluye
%rutas para ciertos comandos, resaltado de sintaxis, previsualización
%del documento, menú matemático en el cual podemos acceder e insertar
%la gran mayoria de los símbolos matemáticos, para no tener que
%memorizarlos. Podemos ver un ejemplo de Emacs + Auc\TeX en la figura
%\ref{auctex}

%\figura{Auctex.png}{scale=0.6}{Emacs + Auc\TeX}{auctex}{h}

%Por ejemplo, para cerrar un entorno $\backslash$\texttt{begin()}, con su
%respectivo $\backslash$\texttt{end()}, utilizaremos el atajo
%\comando{C-c M-]}, para añadir un $\backslash$\texttt{item}, tenemos
%el atajo \comando{C-c C-j}, y así unos cuantos, que una vez que nos
%habituamos a ellos, son bastante cómodos.\\

%Además, es bastante configurable, con indentado automático, corrector
%ortográfico y demás. El fichero adjunto a este documento,
%\emph{conf\_emacs} incluye una configuración con varias de estas
%opciones.

%\section*{Doxygen}

%Realmente, \programa{Doxygen} \cite{website:doxygen} no es una herramienta
%que vayamos a utilizar para realizar documentos \LaTeX{}
%directamente. Sin embargo, para la documentación de código si es
%bastante útil.\\

%Esta herramienta realiza una documentación automática de código
%fuente. Es decir, para nuestro PFC, podemos utilizar para generar la
%documentación de las APIs de nuestras librerías y demás. Puede generar
%esta documentación en varios formatos, y entre ellos, \LaTeX, de forma
%que podemos utilizar ese código generado en nuestra memoria de forma
%automática.

%\section*{GNU Make}

%\programa{GNU Make} es el programa de recompilación y de control de
%dependencias por excelencia. Se puede utilizar para compilar proyectos
%software en diversos códigos, o como en el caso de este documento,
%para compilar documentos \LaTeX{} con diversas opciones.\\

%Para más información \cite{pdf:make}

%\section*{Dia}

%\programa{Dia} es un editor de gráficos vectoriales el cual incluye
%distintas plantillas para distintos tipos de gráficos, como pueden ser
%UML, ERe, diagramas de flujo, esquemas Cisco de red y un larguísimo
%etcétera. Podemos ver el interfaz en la figura \ref{dia}

%\figura{dia.png}{scale=0.6}{Interfaz de Dia}{dia}{h}

%Estos diagramas podemos exportarlos a diversos formatos de imagen
%(\texttt{.png}, \texttt{.eps}, ...) o a formato \texttt{.tex}, como
%vimos anteriormente.

% % % % % % % % % % % % % % % % % % % % % % % % % % % % % % % % % % % % % % % % %
En esta sección hablaremos de las herramientas utilizadas durante el desarrollo de \nombrejuego. Para cada herramienta ofreceremos una pequeña descripción de sus funcionalidades y adjuntaremos las razones por las cuales han sido elegidos para el desarrollo frente a otros de similares características.


\section*{Unity}

\section*{Doxygen}
\programa{Doxygen} \cita{doxygen} es una herramienta de documentación automática de código. Mediante la inclusión de comentarios especiales dentro de los ficheros de nuestro proyecto, es capaz de generar documentación con una apariencia atractiva tanto en formato HTML, como en \LaTeX o muchos otros. Es compatible con los lenguajes C, C++, C\#, Frortran, Java, Objective-C, PHP, Python, IDL y algunos más.

Con \programa{Doxygen} se produce una documentación legible por usuarios (con los conocimientos necesarios) u otros miembros del equipo. Es posible incluir diagramas de colaboración y herencia gracias a su uso de GraphViz. Ha sido elegida como herramienta de documentación de código por su sencillez de uso, limpieza y por su amplia aceptación.

\section*{LaTeX + TeXstudio}
\LaTeX \cita{latex-project} es un lenguaje de marcado y un sistema de creación de documentos especialmente orientado al mundo científico y técnico. El lenguaje \programa{Tex} fue concebido por Donald Knuth durante los 80 y en 1984 Leslie Lamport creó \LaTeX como un framework para trabajar con cartas, libros y otro tipo de textos.

Los resultados que produce \LaTeX son coherentes, ordenados y muy limpios. Si bien aprender su uso puede ser complejo, los resultados son de una enorme calidad si los comparamos con los documentos que producen los editores convenciones como \programa{Microsoft Word} o \programa{LibreOffice Writer}. Por esta calidad y posibilidad de automatizar formatos ha sido elegido \LaTeX para la redacción de la documentación del proyecto.

En un principio se eligió ShareLaTeX \cita{sharelatex} como el editor de la documentación el \LaTeX de \nombrejuego. Pues es un editor online con herramientas colaborativas y que permite compilar en cualquier equipo al guardarse como si fuera una copia de seguridad. Pero ante el nulo uso colaborativo a la hora de la verdad, los diversos problemas de sincronizar los cambios y de compilar cuando la página se satura de usuarios, finalmente hicieron que optara por otro editor. 

En este caso el ganador resultó ser TeXstudio \cita{texstudio}, un editor multiplataformas libre, cuya interfaz es cómoda a la hora de visualizar la estructura de los ficheros que componen la documentación, además de incluir corrector ortográfico y sintáctico, autocompletado, herramientas avanzadas de resaltado y editado de texto, un visor de PDF integrado y otras muchas más características.

\section*{Microsoft Visual Studio Express 2013 + UnityVSExpress}
\programa{Microsoft Visual Studio Express 2013} \cita{visual-studio-express} es un programa de desarrollo en entorno de desarrollo integrado (IDE en inglés) para sistemas operativos Windows desarrollado y distribuido por Microsoft Corporation. Soporta varios lenguajes tales como Visual C++, Visual Basic .NET, y a destacar el usado en \nombrejuego que es C\# aunque actualmente se han desarrollado las extensiones necesarias para muchos otros. Es de carácter gratuito y es proporcionado por la compañía Microsoft Corporation.

Los motivos de su elección son, que a pesar de que \programa{Unity} incluya su propio IDE llamado Monodevelop \cita{monodevelop}, que si bien para trabajos sencillos cumple con las expectativas, a la hora de embarcarnos en un proyecto con mayor carga de código tiene problemas a la hora de navegar entre las extensas líneas de código, abrir ficheros del proyecto manualmente, resultando poco eficiente. Así que teniendo en cuenta que el SO con el que se va a desarrollar es Windows dado que \programa{Unity} aún no ha lanzado su editor para plataformas GNU/Linux, se optó por \programa{Microsoft Visual Studio Express 2013}, uno de los IDE más utilizados en Windows, y por ende, en muchas empresas. Ha sido utilizado para el proyecto por su completa herramienta de depuración, corrector ortográfico, autocompletado, entre otras cualidades que han ayudado a mejorar la productividad.

Además en este caso, se hace uso del plugin \programa{UnityVSExpress} \cita{unityvsexpress} que permite la integración de \programa{Microsoft Visual Studio Express 2013} en los proyectos de \programa{Unity} desarrollados en C\#. Facilitando más aún el trabajo a la hora de redactar el código para \nombrejuego.

\section*{Git + GitHub + Git Shell}
\programa{Git} \cita{git} es un sistema de control de versiones libre enfocado en flujos de trabajo no lineales, integridad de datos y rapidez. Inicialmente fue diseñado y desarrollado por Linus Torvals para el kernel de Linux en 2005, y desde entonces se ha convertido en el sistema de control de versiones más usado para el desarrollo de software. \programa{Git} nos permite contar con una copia de seguridad del código de \nombrejuego en todo momento. Gracias a esta herramienta podemos guardar un historial de todas las versiones de los ficheros, así como deshacer cambios en caso de que fuera necesario. También permite crear ramas con distintas versiones del proyecto y luego fusionarlas en otra. Con \programa{Git} conseguimos acceso al código del proyecto desde cualquier equipo. Además, pone a disposición de cualquier interesado el código fuente de forma sencilla.

Existen otros sistemas de control de versiones como \programa{Subversion} o \programa{Mercurial}. Se ha elegido \programa{Git} para aprender a desarrollar en este sistema de control de versiones, al ser el más extendido y requerido a la hora de trabajar en equipos con varios programadores en un proyecto de software. 

Además de \programa{Git}, se ha elegido la forja \programa{GitHub} \cita{github} que usa dicho sistema de control de versiones. Los motivos son que al ser la forja más popular y reconocida a fecha de hoy, e incluir elementos sociales, permite una mayor difusión del proyecto. \programa{GitHub} también proporciona su propia herramienta para Windows, llamada GitHub Windows \cita{github-windows}, con la que gestionar el control de versiones de manera sencilla y visual para los más novatos. Para los más experimentados y familiarizados de UNIX, incluye la herramienta Git Shell que es una terminal para Windows que permite la utilización de los comandos bash de \programa{Git}. 

Existe otra herramienta llamada también Git Shell \cita{git-for-windows}, siendo esta la oficial usada en el proyecto \programa{Git}, pero por comodidad y sencillez se ha optado por la terminal Git Shell de \programa{GitHub}.

\section*{GraphicsGale FreeEdition}
\programa{GraphicsGale} \cita{graphicsgale} es un editor de gráficos orientados al pixel-art para Windows. Permite la utilización de capas, un sistema de animación (con capas de cebolla para facilitar la creación de animaciones), otras muchas características específicas para el uso del pixel-art tales como el control de la paleta de colores.

Ha sido utilizado para la creación y edición de los gráficos de \nombrejuego con estética pixel-art. Siendo elegido por ser un editor muy completo y potente para este tipo de gráficos.

\section*{GIMP}
\programa{GIMP} \cita{gimp} es el editor de imágenes libre del proyecto GNU, de hecho su nombre es un acrónimo de \emph{GNU image Manipulation Program}. Es multiplataforma y está disponible para Windows, GNU/Linux y Mac OS X. No es comparable a soluciones provativas como \programa{Adobe Photoshop} pero es capaz de realizar operaciones bastante avanzadas de forma sencilla.

Ha sido utilizado en \nombrejuego para crear las transparencias necesarias en las hojas de sprites realizados con \programa{GraphicsGale}. Hemos elegido esta herramienta por contar con una licencia libre, ser lo suficientemente potente para nuestras necesidades y estar disponible en varias plataformas.

\section*{Cacoo}

\section*{Gantter}